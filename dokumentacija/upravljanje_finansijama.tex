\subsection{Slučaj upotrebe: Upravljanje finansijama}

\subsubsection*{Kratak opis}
Finansijski službenik upravlja finansijama pozorišta kroz pregled prihoda i rashoda, korekciju plata zaposlenih i generisanje finansijskih izveštaja.

\subsubsection*{Akteri}
\begin{itemize}
    \item \textbf{Finansijski službenik} -- želi na jednostavan način da upravlja finansijama pozorišta.
\end{itemize}

\subsubsection*{Preduslovi}
\begin{itemize}
    \item Finansijski službenik je uspešno prijavljen na sistem.
\end{itemize}

\subsubsection*{Postuslovi}
\begin{itemize}
    \item Obavljeni su finansijski poslovi.
\end{itemize}

\subsubsection*{Osnovni tok}
\begin{enumerate}
    \item Finansijski službenik bira opciju za upravljanje finansijama.
    \item Sistem prikazuje kontrolnu tablu sa trenutnim stanjem i opcijama: "Pregled transakcija", "Korekcija plata" i "Finansijski izveštaji".
    \item Finansijski službenik bira jednu od ponuđenih opcija.
    \item U zavisnosti od izbora, izvršava se jedan od podtokova (P1, P2 ili P3).
\end{enumerate}
Koraci 2-4 se ponavljaju dok finansijski službenik ne odluči da izađe iz modula za upravljanje finansijama klikom na dugme "Nazad", kada se ceo slučaj završava.

\subsubsection*{Podtokovi}

\textbf{P1: Pregled transakcija}
\begin{enumerate}
    \item Sistem prikazuje formu za pretragu transakcija.
    \item Službenik unosi kriterijume filtriranja (datum, iznos, tip).
    \item Sistem prikazuje listu transakcija koje odgovaraju kriterijumima.
    \item Ako finansijski službenik klikne na određenu transakciju
    \begin{itemize}
        \item Sistem prikazuje detalje izabrane transakcije.
    \end{itemize}
    \item Službenik bira opciju za vraćanje na kontrolnu tablu i slučaj se vraća na korak 2 osnovnog toka.
\end{enumerate}

\textbf{P2: Korekcija plata}
\begin{enumerate}
    \item Sistem prikazuje listu zaposlenih sa trenutnim platama.
    \item Službenik bira zaposlenog.
    \item Sistem prikazuje formu za unos nove plate.
    \item Službenik unosi novi iznos plate i objašnjenje.
    \item Sistem proverava da li je izmena unutar dozvoljenog limita.
    \item Sistem ažurira platu zaposlenog u bazi podataka.
    \item Sistem obaveštava finansijskog službenika da je ažuriranje uspešno.
\end{enumerate}
Koraci 1-7 se ponavljaju dok finansijski službenik ne odluči da izađe iz podtoka klikom na dugme "Nazad", čime se slučaj vraća na korak 2 osnovnog toka.

\vspace{0.5cm}

\textbf{P3: Finansijski izveštaji}
\begin{enumerate}
    \item Službenik bira tip izveštaja (mesečni, kvartalni, godišnji).
    \item Sistem generiše izveštaj na osnovu podataka iz baze.
    \item Službenik bira opciju za vraćanje na kontrolnu tablu.
    \item Sistem pita službenika da li želi da preuzme ili štampa izveštaj.
    \begin{itemize}
        \item Ako službenik izabere preuzimanje $\rightarrow$ izvršava se podtok P4.
        \item Ako službenik izabere štampanje $\rightarrow$ izvršava se podtok P5.
        \item Ako službenik izabere povratak bez preuzimanja ili štampanja $\rightarrow$ slučaj se vraća na korak 2 osnovnog toka.
    \end{itemize}
\end{enumerate}

\textbf{P4: Preuzimanje izveštaja}
\begin{enumerate}
    \item Sistem prikazuje formu za preuzimanje izveštaja.
    \item Službenik bira format (PDF, Excel).
    \item Službenik unosi naziv fajla.
    \item Službenik bira lokaciju za čuvanje fajla.
    \item Sistem čuva izveštaj na izabranoj lokaciji.
    \item Sistem obaveštava službenika da je izveštaj uspešno sačuvan.
    \item Slučaj se vraća na korak 2 osnovnog toka.
\end{enumerate}

\textbf{P5: Štampanje izveštaja}
\begin{enumerate}
    \item Sistem prikazuje formu za štampanje izveštaja.
    \item Službenik bira štampač.
    \item Službenik podešava opcije štampe (broj primeraka, obostrano ili jednostrano štampanje i slično).
    \item Službenik potvrđuje štampanje.
    \item Sistem šalje izveštaj na štampač.
    \item Sistem obaveštava službenika da je izveštaj poslat na štampu.
    \item Službenik preuzima izveštaj sa štampača.
    \item Slučaj se vraća na korak 2 osnovnog toka.
\end{enumerate}

\subsubsection*{Alternativni tokovi}

\textbf{A1: Nema rezultata pretrage}
\begin{itemize}
    \item Ako u koraku 3 podtoka P1 nema transakcija za zadate kriterijume, sistem ispisuje poruku o tome i slučaj se vraća na korak 1 podtoka P1.
\end{itemize}

\textbf{A2: Prekoračenje limita za povišicu}
\begin{itemize}
    \item Ako u koraku 4 podtoka P2 finansijski službenik unese iznos koji prelazi dozvoljeni limit izmene plate, sistem prikazuje upozorenje: ''Povišica prelazi dozvoljeni limit.'' Slučaj se vraća na korak 3 podtoka P2.
\end{itemize}

\textbf{A3: Greška na štampaču}
\begin{itemize}
    \item Ako u koraku 5 podtoka P5 dođe do greške prilikom komunikacije sa štampačem, sistem prikazuje poruku o grešci i slučaj se vraća na korak 1 podtoka P5.
\end{itemize}

\subsubsection*{Specijalni zahtevi}
\begin{itemize}
    \item Sistem mora automatski beležiti ko je i kada izvršio promenu plate.
\end{itemize}

\subsubsection*{Dodatne informacije}
\begin{itemize}
    \item Limit za izmenu plate je propisan zakonom.
\end{itemize}

\begin{figure}[h!]
    \centering
    \includegraphics[width=\textwidth]{./Slučajevi upotrebe/upravljanje_finansijama.png} 
    \caption{Dijagram slučaja upotrebe "Upravljanje finansijama"}
\end{figure}

\newpage
