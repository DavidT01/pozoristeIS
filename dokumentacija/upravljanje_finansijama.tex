\subsection{Slučaj upotrebe: Upravljanje finansijama}

\subsubsection*{Kratak opis}
Finansijski službenik upravlja finansijama pozorišta kroz pregled prihoda i rashoda, korekciju plata zaposlenih i generisanje finansijskih izveštaja.

\subsubsection*{Akteri}
\begin{itemize}
    \item \textbf{Finansijski službenik} -- želi na jednostavan način da upravlja finansijama pozorišta.
\end{itemize}

\subsubsection*{Preduslovi}
\begin{itemize}
    \item Finansijski službenik je uspešno prijavljen na sistem.
\end{itemize}

\subsubsection*{Postuslovi}
\begin{itemize}
    \item Obavljeni su finansijski poslovi.
\end{itemize}

\subsubsection*{Osnovni tok}
\begin{enumerate}
    \item Finansijski službenik bira opciju za upravljanje finansijama.
    \item Sistem prikazuje kontrolnu tablu sa trenutnim stanjem i opcijama: "Pregled transakcija", "Korekcija plata" i "Finansijski izveštaji".
    \item Finansijski službenik bira jednu od ponuđenih opcija.
    \item U zavisnosti od izbora, izvršava se jedan od podtokova (A, B ili C).
\end{enumerate}

\subsubsection*{Podtokovi}

\textbf{A: Pregled transakcija}
\begin{enumerate}
    \item Sistem prikazuje formu za pretragu transakcija.
    \item Službenik unosi kriterijume filtriranja (datum, iznos, tip).
    \item Sistem prikazuje listu transakcija koje odgovaraju kriterijumima.
\end{enumerate}

\textbf{B: Korekcija plata}
\begin{enumerate}
    \item Sistem prikazuje listu zaposlenih sa trenutnim platama.
    \item Službenik bira zaposlenog.
    \item Sistem prikazuje formu za unos nove plate.
    \item Službenik unosi novi iznos plate i objašnjenje.
    \item Sistem proverava da li je povišica unutar dozvoljenog limita.
    \item Sistem ažurira platu zaposlenog u bazi podataka.
    \item Sistem obaveštava finansijskog službenika da je ažuriranje uspešno.
\end{enumerate}

\textbf{C: Finansijski izveštaji}
\begin{enumerate}
    \item Službenik bira tip izveštaja (mesečni, kvartalni, godišnji).
    \item Sistem generiše izveštaj na osnovu podataka iz baze.
    \item Sistem omogućava preuzimanje ili štampanje izveštaja.
\end{enumerate}

\subsubsection*{Alternativni tokovi}

\textbf{A1: Nema rezultata pretrage}
\begin{itemize}
    \item Ako u koraku 3 podtoka A nema transakcija za zadate kriterijume, sistem ispisuje poruku o tome i slučaj se vraća na korak 1 podtoka A.
\end{itemize}

\textbf{B1: Prekoračenje limita za povišicu}
\begin{itemize}
    \item Ako u koraku 4 podtoka B finansijski službenik unese iznos koji prelazi dozvoljeni limit povišice, sistem prikazuje upozorenje: ''Povišica prelazi dozvoljeni limit.'' Slučaj se vraća na korak 3 podtoka B.
\end{itemize}

\subsubsection*{Specijalni zahtevi}
\begin{itemize}
    \item Sistem mora automatski beležiti ko je i kada izvršio promenu plate.
\end{itemize}

\subsubsection*{Dodatne informacije}
\begin{itemize}
    \item Limit za povišice je propisan zakonom.
\end{itemize}

\begin{figure}[h!]
    \centering
    \includegraphics[width=\textwidth]{./Slučajevi upotrebe/upravljanje_finansijama.png} 
    \caption{Dijagram slučaja upotrebe "Upravljanje finansijama"}
\end{figure}
