\section{Slučaj upotrebe: Pregled rashoda}

\subsection*{Kratak opis}
Finansijski službenik pregleda sve rashode pozorišta, uključujući plate zaposlenih, uplate za opremu i druge troškove.

\subsection*{Akteri}
\begin{itemize}
    \item \textbf{Finansijski službenik} -- želi da prati finansijske troškove pozorišta.
\end{itemize}

\subsection*{Preduslovi}
\begin{itemize}
    \item Finansijski službenik je uspešno prijavljen na sistem.
\end{itemize}

\subsection*{Osnovni tok}
\begin{enumerate}
    \item Finansijski službenik bira opciju za pregled rashoda u glavnom meniju.
    \item Sistem prikazuje formu za filtriranje rashoda.
    \item Finansijski službenik unosi željene kriterijume.
    \item Sistem pretražuje bazu podataka prema zadatim parametrima.
    \item Sistem prikazuje listu pronađenih rashoda.
    \item Finansijski službenik može odabrati pojedinačni rashod za detaljniji prikaz.
    \item Sistem prikazuje sve detalje odabranog rashoda.
\end{enumerate}

\subsection*{Postuslovi}
\begin{itemize}
    \item Finansijski službenik je informisan o stanju rashoda.
\end{itemize}

\subsection*{Alternativni tokovi}

\textbf{A1: Nema rezultata pretrage}
\begin{itemize}
    \item Ukoliko za zadate kriterijume ne postoje rashodi, sistem obaveštava službenika porukom "Nema pronađenih rashoda za zadate kriterijume". Finansijski službenik se vraća na korak 2 i može izmeniti kriterijume pretrage.
\end{itemize}

\subsection*{Dodatne informacije}
\begin{itemize}
    \item U koraku 2 službenik može filtrirati po datumu, vrsti troška, iznosu i primaocu.
    \item U koraku 5 sistem prikazuje naziv, datum, vrstu i iznos rashoda, kao i primaoca.
\end{itemize}

\begin{figure}[h!]
    \centering
    \includegraphics[width=\textwidth]{./dijagrami/pregled_rashoda.png} 
    \caption{Dijagram slučaja upotrebe "Pregled rashoda"}
\end{figure}
