\section{Slučaj upotrebe: Evidencija prihoda}

\subsection*{Kratak opis}
 Sistem automatski beleži finansijske detalje transakcije u centralnu bazu radi finansijskog izveštavanja.

\subsection*{Akteri}
\begin{itemize}
   \item \textbf{Sistem} -- izvršava proces automatski.
    \item \textbf{Finansijski službenik} -- ima uvid u evidentirane podatke (pasivan učesnik u trenutku izvršenja).
\end{itemize}

\subsection*{Preduslovi}
\begin{itemize}
  \item Roditeljski slučaj upotrebe (npr. Prodaja karata) je uspešno autorizovao plaćanje.
    \item Podaci o transakciji su dostupni.
\end{itemize}

\subsection*{Osnovni tok}
\begin{enumerate}
    \item Sistem preuzima parametre transakcije iz roditeljskog procesa (iznos, datum, vreme, način plaćanja).
    \item Sistem klasifikuje prihod prema načinu plaćanja (Gotovina / Kartica / Online).
    \item Sistem upisuje transakciju u Dnevnik prihoda (Ledger).
    \item Sistem ažurira trenutno stanje blagajne (ako je gotovina) ili virtuelnog računa (ako je kartica).
    \item Sistem generiše jedinstveni ID transakcije za potrebe revizije.
    \item Slučaj upotrebe se završava i kontrola se vraća roditeljskom procesu.
\end{enumerate}

\subsection*{Postuslovi}
\begin{itemize}
    \item Transakcija je trajno sačuvana u finansijskoj bazi podataka.
    \item Finansijski izveštaji su ažurirani novim iznosom.
\end{itemize}

\subsection*{Alternativni tokovi}

\textbf{A1: Greška pri upisu u bazu}
\begin{itemize}
    \item Ukoliko baza podataka nije dostupna, Sistem privremeno čuva podatke o transakciji u lokalni log fajl (cache).Sistem šalje notifikaciju Administratoru o grešci.
 Sistem vraća potvrdu roditeljskom procesu da je prodaja uspela (kako ne bi blokirao kupca), uz napomenu da će evidencija kasniti. Po uspostavljanju veze, sistem naknadno sinhronizuje podatke.
\end{itemize}

