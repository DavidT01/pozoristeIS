\section{Uvod}
Informacioni sistem pozorišta je softversko rešenje dizajnirano da modernizuje i olakša upravljanje svim aspektima rada pozorišta. Sistem integriše ključne poslovne procese, uključujući upravljanje predstavama i repertoarom, prodaju karata (online i na blagajni), organizaciju audicija, evidenciju tehničkih potreba i opreme, kao i finansijsko poslovanje i administraciju zaposlenih. Cilj sistema je povećanje efikasnosti, transparentnosti i dostupnosti informacija svim akterima, od uprave i zaposlenih do publike.

\section{Akteri}
Sistem prepoznaje sledeće grupe korisnika koji interaguju sa funkcionalnostima na različite načine:

\begin{itemize}
    \item \textbf{Reditelj}: Odgovoran za umetnički deo, kreiranje predstava, odabir glumaca na audicijama i definisanje potreba predstave.
    \item \textbf{Glumac}: Umetnik koji učestvuje u audicijama i realizaciji predstava.
    \item \textbf{Administrator}: Zadužen za tehničku administraciju sistema, unos novih zaposlenih, ažuriranje informacija o predstavama i zakazivanje termina.
    \item \textbf{Tehničko osoblje}: Brine o tehničkim resursima, evidentira potrebe za opremom i vrši popravke u pozorištu.
    \item \textbf{Finansijski službenik}: Upravlja budžetom, odobrava troškove, naručuje opremu, isplaćuje plate i generiše finansijske izveštaje.
    \item \textbf{Gledalac}: Korisnik usluga pozorišta koji pretražuje repertoar i kupuje karte putem interneta ili na blagajni.
    \item \textbf{Blagajnik}: Zaposleni koji vrši direktnu prodaju karata na prodajnom mestu u pozorištu.
\end{itemize}